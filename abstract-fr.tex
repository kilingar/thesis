\renewcommand{\abstractname}{R\'esum\'e}
\begin{abstract}
\addcontentsline{toc}{chapter}{R\'esum\'e}
\markboth{R\'esum\'e}{R\'esum\'e}
Dans ce travail, un générateur de volumes élémentaires représentatifs (VER) basé sur les champs de distance d'un agrégat d'inclusions de forme arbitraire est développé dans le cadre de matériaux moussés à structure ouverte. Lorsque les inclusions sont sphériques, la tessellation de l'agrégat résulte en des morphologies similaires aux échantillons de mousse physique en termes de rapports des nombres de face par pores et de bords par faces, ainsi que de la distribution de la longueur des entretoises, entre autres. Les fonctions qui combinent les champs de distance peuvent être utilisées pour obtenir des tesselations avec les variations nécessaires aux géométries des entretoises et extraire ces morphologies de mousse ouverte. Il est également possible de remplacer l'agrégat d'inclusions par un ensemble prédéfini d'inclusions qui sont directement extraites d'images tomographiques.

L'utilisation de fonctions de niveaux discrètes entraîne de fortes discontinuités dans les dérivées des champs de distance. Une approche basée sur des ensembles de niveaux multiples est présentée qui peut capturer de manière appropriée les arêtes vives des entretoises des mousses ouvertes à partir des champs de distance résultants. Une telle approche peut contourner les discontinuités présentées par les champs de distance qui pourraient conduire à des concentrations de contraintes parasites dans une analyse du
comportement des matériaux.

Les pores individuels sont ensuite extraits en tant que surfaces d'inclusions sur la base desdites combinaisons des fonctions de distance et de leurs modifications. Ces surfaces peuvent être réunies pour obtenir la géométrie finale des morphologies de mousse ouverte. Les attributs physiques des géométries extraites sont comparés aux données expérimentales. Une comparaison statistique est présentée décrivant les différentes caractéristiques. L'étude est étendue aux morphologies qui ont été extraites à l'aide d'images tomographiques.

À l'aide d'outils d'optimisation de maillage, les triangulations des surfaces peuvent être obtenues, fusionnées et développées sous forme de modèles d'éléments finis (FE). Les modèles sont prêts à être utilisés dans une étude multi-échelle pour obtenir le comportement homogénéisé du matériau. La mise à l'échelle peut aider à évaluer les applications pratiques de ces modèles en les comparant aux données expérimentales d'échantillons physiques. Le comportement des matériaux des VERs est également comparé aux observations expérimentales.

Pour augmenter l'efficacité de calcul de l'étude, un modèle de substitution basé sur un réseau neuronal est présenté. Ce modèle peut remplacer le problème aux valeurs limites à l'échelle micro dans une analyse multi-échelle. Les réseaux de neurones sont construits à l'aide de modules spécialement conçus pour prédire le comportement dépendant de l'histoire et sont appelés réseaux de neurones récurrents (RNN). Les modèles de substitution sont entrainés pour prendre en compte le caractère aléatoire du chargement que subit un matériau complexe lors d'une analyse de comportement d'un matériau.
\end{abstract}