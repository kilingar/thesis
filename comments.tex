Proposals for modification of the PhD thesis manuscript of
N. Kilingar

The following proposals for modification were formulated by the jury during the private PhD
defence (7 th December 2020, 9 AM – 12:45 PM through videoconference). They are split
into compulsory and optional groups. While applying the compulsory modifications (noted
CM) is a prerequisite for the public PhD defence (they will be checked for in the revised
document), the optional modifications would contribute to enhancing the quality of the
manuscript as well in the jury’s opinion.

Complying with the jury’s decision after deliberation on 7 th December 2020, the candidate
has to prepare a revised version of the PhD manuscript and submit it to the jury for
approval before the public defence takes place. Two versions of the revised manuscript
are to be sent: one in which the applied modifications are highlighted in a different color (to
allow an efficient verification of the modifications) and a version ready to submit (text
written entirely in black). These documents are to be submitted by 10 th January 2021.

General CM

GCM1 – reproducibility: ensure that you give systematically all information
required for an interested reader to be able to reproduce your results in all chapters
of the manuscript
%TODO

GCM2 – transparency: give systematically a recommendation (this ideally could be
expressed as an algorithm to follow) on how to perform all of the trial-error based
procedures you used in the work
%TODO

Chapter 1-2

CM1: add a clear identification of gaps in the literature and the proposal of
corresponding actions in the thesis and their originality with respect to the state of
the art (SoA) - this can be formatted as a list; highlight personal original
contributions with respect to a well-defined starting point (i.e. the building blocks
that were available at the start); recall the specific personal work and contribution to
the state of the art in each section
%TODO

explain what size effects are present in the mechanical behavior of foams by an
illustration of how the foam sample response is expected to change as a function of
the sample size

CM2: define the limitations of computational homogenization vs localized
phenomena in the foams (buckling, localized plastic strains)
%TODO Ludovic notes

CM3: establish a link between second order homogenization and higher order
continua; justify using a first order approach vs observed foam behavior (no
softening, at least when in 3D)
%TODO Intro chap 2

when defining an RVE explain that its definition depends on the physics that is
aimed at, i.e. for mechanical behavior and for acoustic insulation they are different

Chapter 3-4

CM4: state clearly what are your own developments for geometry generation (wrt B.
Sonon) and meshing (wrt K. Ehab) in a dedicated paragraph
%TODO Updated in the Introduction of chapter 3

CM5: explain why a statistical/stochastic description is needed in your work and
how an “ideal geometry” would perform compared to this
%TODO

CM6: critically discuss the limitations of the assumptions you use for using a
(periodic) RVE vs experimentally observed behavior of foams (i.e. strain
localization); state the domain of validity and the domain of application of the
simplifications you make
%TODO Chapter 2 Intro?

give a quantitative recommendation of how to choose the number of grid points for
a given number of inclusions based on your experience (GCM2)

CM7: list all sources of randomness that you incorporate in your model as well as
the ones that are neglected (e.g. position along the length of struts where the cross
section is minimal); justify your choice of the set of parameters you drive and
classify the ones you disregard in terms of potential criticality on the foam
mechanical response
%TODO

CM8: explain in a dedicated paragraph more clearly how the presence of cell walls
is triggered (open/closed cell decision)
%TODO Explanation in Section Open / close faces

indicate how the geometry generation parameters can be obtained for a given real
foam sample based on your experience (GCM2)

mention that computational methods other than FEM could be used as well with
your approach (e.g. X-FEM)

CM9: explain briefly how your formulation would allow considering beams
%TODO In edge description

explain how the case of hollow cross sections of struts can be considered

express more clearly how the mesh optimization is performed (moving nodes, but
keeping the same mesh topology); comment if this approach is good enough for
large strain analyses and how it could be further improved (GCM1), also by
verifying how refinement & edge flips were implemented by K. Ehab

comment in more details on why the FEM response is considered to be a good
match to experiments

mention as outlook that the individual effect of each incorporated morphological
parameter could be investigated in the future to assess their criticality in a numerical
model

Chapter 5

CM10: state clearly and concisely your specific personal contributions and how they
advance the SoA of NN
%TODO Introduction

sufficient details on the validation technique, choice of the optimization parameters
and validation sets are missing; the reproducibility of your methodology and work
needs to be insured (GCM1)

refer to other methods than FE 2 that you could use to incorporate microstructural
information based on a rapid sweep of the literature

CM11: establish with more clarity the link between FE 2 and NN, including the
motivation and advantages of the NN approach
%TODO Introduction of NN

CM12: mention what other surrogate models could have been used (non-
exhaustive list with references) and their dis/advantages with respect to NN
%TODO Introduction of NN

CM13: critically discuss in a paragraph whether NN can be intended to be
employed to find a fully upscaled constitutive behavior for general loading and a
fully general material behavior; define the context and limitations of your work with
respect to this discussion; define the range of applicability of the proposed
approach (and maximum applied strains) vs occurrence of localized phenomena
%TODO Conclusion

explain how you control the validation error vs overfit and give a recommendation to
the interested reader; give a recommendation allowing to choose the number of
epochs; explain your “manual” choice and how to reproduce it by an interested
reader (GCM2)

comment on how you dealt with possible local minima appearing in the parameter
optimization of the NN and if there are automated methods available for this
(GCM2)

explain what “stochastic” refers to in technical terms in the stochastic gradient
descent approach

explain how do you define the number of nodes and number of layers; give a
recommendation of how an interested reader could make an informed choice;
explain what hyperparameters are and how do you perform your trial-error
procedure in obtaining them (GCM2)

CM14: clarify that clusters are chosen in the beginning of the simulation and kept
fixed during the simulation; explain how they are chosen and comment on the fixed
clusters assumption vs evolving stresses and possible internal force redistribution;
mention what other approaches may be available for clustering; what quantities are
clustered and why?
%TODO Section clustering

explain how the FNN 2 approach approximation could be enhanced, e.g.
simultaneous optimization of both FNNs, instead of the current sequential process
to decrease the accumulating error

CM15: give the order of magnitude of the prediction time of your networks; when
comparing the computational time for NN setup to the equivalent number of FE
simulations, give an example of what size of FE 2 simulation this number would
correspond to and highlight the advantages of the NN approach (the same trained
network can be used for any simulation with the same base material)
%TODO Computstional cost

Chapter 6

CM16: reformulate and strengthen the conclusions with respect to the initial goals
and identified gaps in the literature instead of a summary of each chapter; highlight
the advancement of the SoA due to your contribution
%TODO Comclusions

CM17: more details are expected in the outlook of the work
%TODO
