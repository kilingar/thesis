\documentclass[aspectratio=169,11pt]{beamer}
\usepackage[latin1]{inputenc}
%\usepackage[T1]{fontenc}
\usepackage[english]{babel}
\usepackage{amsmath}
\usepackage{amsfonts}
\usepackage{amssymb}
\usepackage{graphicx}
\usetheme{Singapore}
\begin{document}
	\author[Kilingar]{Nanda Gopala Kilingar}
	\title[]{Generation and data-driven upscaling of
		open foam representational volume elements}
	\subtitle{Notes and preparations}
	%\logo{}
	\institute{Universit\'e de Li\`ege\\ Universit\'e Libre de Bruxelles}
	\date{}
	%\subject{}
	%\setbeamercovered{transparent}
	%\setbeamertemplate{navigation symbols}{}
	\begin{frame}[plain]
	\maketitle
\end{frame}

\section{Introduction}
\begin{frame}
\frametitle{Introduction to open foam}
\begin{itemize}
\item Methodologies to obtain numerical models (Symmetric models, tessellations, surface evolver, image processing)
\item Morphological properties with some figures Redenbach 2013
\item Behavioral properties with data from Jung 2017 and Heinze 2018
\end{itemize}
\end{frame}

\begin{frame}{Objectives and Methodologies}
\scriptsize{
\begin{block}{Aim}
	\begin{itemize}
		\item Fast generation of random open foam morphologies using distance fields based inclusion packings.
		\item Use CT-scan image based data to obtain inclusion packings.
		\item Implement surrogates to replace complex micro-structures in expensive computational homogenization techniques.
	\end{itemize}
\end{block}
\begin{block}{Means}
\begin{itemize}
	\item Conceptualize level set functions to obtain geometrical variations in morphologies that manipulate the distance fields.
	\item Generate mesh using refinement tools that respect the distance field based morphological constraints.
	\item Compare morphological indicators with literature to develop possible ranges to the function constraints.
	\item Analyze numerical behavior of the random morphologies and the CT-scan based morphologies and compare with experiments.
	\item Study Neural network based surrogates to understand the practical constraints of machine learning methods when applied to real world mechanical problems.
\end{itemize}
\end{block}
}
\end{frame}

\section{Open foam morpholgies}
\begin{frame}{Packing generation}
	\begin{columns}[onlytextwidth]
		\column[]{0.45\textwidth}
		\begin{block}{Random Packing}
			\begin{itemize}
				\item Generate inclusion packing based on packing indicators from literature
				\item Generate plateau level set ($ O_P $) using the distance fields and nearest neighbor information
			\end{itemize}
		\end{block}
		\column[]{0.45\textwidth}
		\begin{block}{CT-scan packing}
			\begin{itemize}
				\item Extract CT-scan images from physical foam samples
				\item Obtain the constituent inclusions forming the foam pores in terms of ellipsoids or polyhedrals
				\item Generate distance fields based on the extracted inclusions
			\end{itemize}
		\end{block}
	\end{columns}
\end{frame}

\begin{frame}{Sharp Edges}
	\begin{itemize}
		\item .
	\end{itemize}
\end{frame}

\begin{frame}{Mesh Refinement}
\begin{itemize}
	\item .
\end{itemize}
\end{frame}

\section{Morphology Analysis}
\begin{frame}{Morphology Quantification}
\begin{itemize}
	\item .
\end{itemize}
\end{frame}

\begin{frame}{Finite element modeling}
\begin{itemize}
	\item .
\end{itemize}
\end{frame}

\begin{frame}{Conclusions \#1 and \#2}
\begin{itemize}
	\item .
\end{itemize}
\end{frame}

\section{Neural network models}
\begin{frame}{Neural networks as surrogates}
\begin{itemize}
	\item .
\end{itemize}
\end{frame}

\begin{frame}{FNNs and RNNs}
\begin{itemize}
	\item .
\end{itemize}
\end{frame}

\begin{frame}{Study of predictions}
\begin{itemize}
	\item .
\end{itemize}
\end{frame}

\begin{frame}{Conclusions \#3}
\begin{itemize}
	\item .
\end{itemize}
\end{frame}

\section{Conslusions}
\begin{frame}{Summary and perspectives}
\begin{itemize}
	\item .
\end{itemize}
\end{frame}

\begin{frame}{References}
\begin{itemize}
	\item .
\end{itemize}
\end{frame}

\end{document}