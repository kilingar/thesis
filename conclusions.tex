% Draft Version 3 - Date 27/08/2020
% Draft Version 4 - Date 14/12/2020
\chapter{General Conclusions and perspectives}

The accurate representation of the behavior of materials is important to predict the longevity and the possible failure of structures made of these materials. In this regard building representative models based on finite elements has been a boon to engineers and the added improvement in the performance of modern processors has only increased the expectation of the prediction of the behavioral pattern.

In this regard, this thesis presents an innovative approach to model complex micro-structures that porous metallic materials display in general and makes an attempt at upscaling the behavior using innovative machine learning models based on deep neural networks. The aim of this study is to improve the computational effort and the time taken to analyze complex geometrical behavior without losing the accuracy of the predicted values.

A computational homogenization framework has been recalled in Chapter \ref{chap-ch} that summarizes the various aspects of a multi-scale analysis starting from the micro-scale to the macro-scale with the necessary boundary conditions to satisfy the equilibrium conditions. Such a framework needs an RVE that is statistically representative of the general geometry of the material. 

\red{In Chapter \ref{chap-of}, we try to fulfill this need with the help of random packings and extracting the resultant distance fields to construct level sets with various attributes implemented with the help of the extracted distance fields. The tool used here, DN-RSA, is capable of generating anisotropy, strut cross-section variations, and closed/open faces of the pores among other useful features. The details of open foam struts geometries was not presented in \cite{sononAdvancedApproachGeneration2015}. In this regard, the extraction of really accurate sharp edges of the foam strut with specific geometry of the strut cross-section was implemented in this work. We move further ahead and claim that the tool, with certain modifications implemented in it, can be then used along with CT-scans of the porous metallic structures to generate highly accurate morphologies. We have aptly renamed this modified tool as DN-CT-SCAN. The capabilities are further enhanced because of the nature of the final output of this tool, a finite element model of the geometry that can take into account the intricate details of the constituent strut shapes in the morphologies. }

The basic idea behind the generation of these open foam morphologies is that the tessellations of inclusions in a closed packing will display behavior similar to that of the physical foam samples considering that the foam samples are manufactured when the surface tension reduces sufficiently and perforations in the material is observed leading to the opening of the pore walls. Various studies have shown these existing similarities and comparisons have also been drawn using numerical tools to implement this minimization of surface energy \cite{kraynikStructureRandomFoam2004}. 

To display the geometrical as well as behavioral aspects of the generated morphologies, a detailed analysis has been presented in Chapter \ref{chap-res}. This analysis also presents the state-of-the-art in the field of morphology extraction using tessellations based analysis and shows the computational advantage in using DN-RSA and related tools to obtain the open foam morphologies. With a selected few important features like the distribution of the strut lengths, the number of faces to pore ratio and the number of edges per face ratio, one can summarize the most important aspects of the behavior as these are the parameters that influence the occurrence of local plasticity in the morphology. With the help of CT-scans made available to us by our collaborators in Saarland University, we were able to display the geometrical closeness of the morphology extracted using DN-CT-SCAN with the foam sample. 

To improve the versatility of the RVE generation tool, a relation has been established between the pores per inch (ppi) of physical foam samples and DN-RSA since ppi is the commercial property that decides the selection of a particular foam. The variation of ppi was collected from various literature survey and a general relation has been established by varying the strut cross-section along its axis and also at the mid-axis. The visual results of implementing these features have also been presented further establishing the similarity of the extracted RVE with experimental observation. This similarity motivated the study using finite element modeling of the extracted geometries.

A simple initial study on individual pores establishes what was already observed using experimental tools and confirmed the matching behavior of the pores taken from randomly extracted RVE with that of physical samples as observed in \cite{heinzeExperimentalNumericalInvestigation2018}. The prepared random RVEs were then subjected to the same loading using numerical tools as the experimental ones in \cite{jungMicrostructuralCharacterisationExperimental2017}. For the purpose of representation, Figure \ref{25-stress-strain} shows the homogenized behavior of the random RVEs along with the experimental data where the RVEs were generated using established relations with the aim of obtaining 20ppi morphologies. Further, the behavior of the morphology generated using DN-CT-SCAN was also analyzed. Again, the homogenized behavior of the geometries extracted geometries mimic very closely the pattern observed during experimental analysis, especially for the morphologies extracted from basis polyhedrals in the elastic regime and the initial plateau-like behavior.

\red{The morphology generation tool, presented here, is very fast and efficient in comparison to existing tools that deal with porous geometry. One can easily obtain an RVE made of more than 100 pores in under an hour, including any possible manual post-processing, with the final finite element mesh having anywhere upto 500,000 elements. We have found this process to also be computationally less expensive with the added advantage of the end product having a refined morphology. The tool can be plugged in to generate multiple such random RVEs on-line when dealing with a large scale computational homogenization procedure due to its ability to start from nothing but the initial sphere size and generate FE models that can be quickly inserted in the \fee framework.}

The applicability of various boundary conditions on the extracted non-periodic geometries has also been presented. A comprehensive view of the use of existing algorithms to impose BCs like periodic and mixed on such geometries shows the effect smaller RVEs may have in the interpretation of the obtained behavior. Nevertheless, the existing CT-scan technology for image extraction, at least for metallic foams, does not permit the study of larger structures and this handicap might be overcome in the future by other methods to obtain foam images. But we believe that as long as these images can be used to extract the core basis inclusions that resulted in the eventual physical foam, DN-CT-SCAN will always be one of the quickest tool to efficiently obtain a close morphology that can then be used for various types of numerical studies.

This study acknowledges the existence of hybrid metallic foams\cite{jungHybridMetalFoams2014} and the presented tool can be used to extract morphologies mimicking these hybrid foams that have various coatings over the core foam. This procedure may not be time consuming due to the existence of pre-saved distance fields that need simple modifications to appropriately identify the level sets corresponding to the various coatings. A study can be done in the future that can fine tune the parameters necessary to implement these level sets and compare the eventual material behavior of the resultant foams with physical samples.

\red{To make the tool a true \textit{one-click} algorithm, the robustness of the meshing process needs to be looked at. To ensure local geometrical topology, such as the open / close faces or the varying thickness of the struts, can be well captured, a control needs to be implemented to vary the level set functions and eventually the mesh size around such phenomena. }

To ease the effort necessary to compute the material response of such complex geometries, we have made an attempt at conceptualizing a machine learning based upscaling scheme in Chapter \ref{chap-nn} that uses artificial neural networks that can be used as a surrogate for the micro-scale BVP in the larger multi-scale analysis. The state-of-the-art in the field of machine learning has been presented before delving deeper into the methodologies related to neural networks that we have used to constitute this study.

To ensure that the one-is-to-one relation presented by a simple monotonic proportional loading of material presents with respect to homogenized stress-strain states, a general feedforward neural network has been presented, trained and validated for this case study and the test results have been presented. The study has then been expanded on to complicated loading patterns generally observed in materials such as repeated loading-unloading behavior and local random loading behavior. For this purpose recurrent neural network modules have been used because of its use of the underlying state dependent behavior. The state dependency has also been replicated using a double feedforward network based model and the two types of models have been compared. 

The state-based representation used in the proposed \fnn method underscores the utility of the intermediate incremental strain-state both in its utility to predict the homogenized stress states as well as its use, with simple relations, to continue the DNS analysis from any given strain state. 

The neural network study has established a new approach to analyze material behavior owing to its promising results and the further improvements in the artificial neural network modules that can be expected in the future only necessitates that these studies are well documented to be used as a precursor. However, at present such models can only be generated on a trial basis, and can be used only at a very personal level with modifications necessary with changes in the type of the material and the behavior it presents to various loading cases.

%Overall, this work presents an efficient and fast tool to generate micro-structures of porous materials that can be used to extract morphologies that are not only statistically close to the physical samples but are also ready to use in a finite element framework. A single morphology has been used to study the behavior under multiple loading conditions using neural networks.

\red{The future scope for this study is vast. The computational homogenization procedure explained follows a first order continuum. To capture localization phenomena, higher order or micromorphic simulations need to be conducted for training the ANNs along with the higher order deformation gradient. Notwithstanding the computational effort necessary, the neural network study has to be expanded on to 3D morphologies and the complexity presented by the added dimensions needs to be analyzed. Due to the time-constraints, a study on the use of neural networks with multiple geometries has not been presented here, but this is an obvious area that can be investigated either using networks that make use of image analysis or using the internal state variable or its clusters. Continuing with the state-based representation, one can add these features to the RNN-modules or design another intermediate FNN model to go with the \fnn model and possibly predict the future behavior of the material under any possible loading situation. Also, the speed with which machine learning algorithms are being developed owing to the ease of computational effort, the neural networks and the related data driven models can be used to study experimental foam behavior by tweaking out the instabilities and uncertainties occurring due to machine errors by identifying these features with ease.}

The field of machine learning, though not new, has only opened up recently on a wide scale for experimentation and improving the predictions of mechanical behavior and there is definitely a lot to explore and learn. However, it is our belief that with the current capabilities of neural networks, FEM based DNS is still necessary as there is too much trial and error involved in the data driven models and can not be reliably used as plug-ins without keeping them under constant observation during the analysis.
